% Ein -*-latex-*- File

\songtitle{Unicards}
\begin{songinfo}
Melodie: Die Moorsoldaten (R. Goguel)\\
Text: Stony \& KaiN Jan96\\
Enstanden auf der Fahrt zum Redaktionstreffen f�r den
Chipkartenreader%
\footnote{AK Chipkarten, KIF Hamburg:
  \texttt{\mbox{http://www.igd.fhg.de/\~{}kumpf/AKCK/}}}
\end{songinfo}

\begin{song}

W\X{em}ohin auch das Auge blicket
   K\X{H7}artenleser \X{em}steh\X{[H7]}n her\X{em}um\\*
D\X{G}atenflut, die uns ersticket,
   doch die St\X{H7}udis \X{em}blei\X{[H7]}ben st\X{em}umm

\Xvspace

\textbf{Refrain (2x):}\\*
\X{D}---  St\X{G}udis zieh�n mit K\X{D}arten,
   d\X{am}ie sie voll verd\X{H7}aten --- vor\X{em}an

\Xvspace

T�ren werden kontrolliert, Sicherheit ist angesagt\\*
Ein- und Ausgang registriert, ohne da� man jemand fragt

\textbf{Refrain}

Mensa, Wohnheim, Bibliotheken, nirgendwo mehr anonym\\*
Privat, vertraulich, unbewacht und kriminell sind synonym

\textbf{Refrain}

Willst du mal Beamter werden, meide B�cher mit Kritik\\*
Denn dein Dienstherr mi�t Gesinnung dank Biblio-Card auf einen Blick

\textbf{Refrain}

Wie oft warst du in der Kirche, in der K�che und im Bad?\\*
Fr�her wu�te es nur der Nachbar, mit dem Fernglas --- heut der Staat

\textbf{Refrain}

Auch die Wirtschaft zieht aus Daten ihre Marktinformation\\*
Aufbereitet in der Werbung dient sie der Manipulation

\textbf{Refrain}

\end{song}

\vfill

\begin{quote}
  Als der BND Telefonate nach Schl�sselw�rtern absuchte, schwieg ich,
  ich hatte mir ja nichts zuschulden kommen lassen.\\
  Auch als wegen Newsgroups ermittelt werden sollte, schwieg ich, ich
  hatte ja nur unverf�ngliche Sachen geschrieben.\\
  Doch als das Internet von Staats wegen zensiert wurde, schrie ich
  auf. Aber Rundfunk wie Zeitung waren zensiert, B�cher wurden
  verbrannt, und so konnte mich niemand mehr h�ren.
  \mbox{}\hfill\textit{(unknown)}
\end{quote}

\vfill

