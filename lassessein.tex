% Ein -*-latex-*- File

\songtitle{La� es sein}

\begin{songinfo}
Melodie: Let It Be (John Lennon, Paul McCartney)%
\footnote{F�r die TechnoGeneration: Die haben was mit den Beatles zu
  tun, das war mal eine sehr popul�re Band}\\
Text: Wei� nicht genau\\
Ich hab das so �hnlich auf einer Party aufgeschnappt und aus dem
Ged�chtnis zusammengebastelt, keine Ahnung, von wem das urspr�nglich
stammt. (Stony)
\end{songinfo}

\begin{song}
Ich s\X{C}itz im Fernsehs\X{G}essel und sch\X{am}enk mir grad \X{[G]}ein 
       B\X{F}ierchen ein,\\
Doch d\X{C}a kommt meine Fr\X{G}au und sagt: 
       "`La� es s\X{F}ein!"'\X{[C dm7] C}\\
Ich h\X{C}ol mein T�tle T\X{G}abak und r\X{am}auch 'ne 
       Z\X{[G]}igar\X{F}ette klein,\\
Meine Fr\X{C}au steht in der T\X{G}�re: "`La� es s\X{F}ein!"'\X{[C dm7] C}

\Xvspace

\textbf{Refrain:}\\
L\X{[G]}a� es s\X{am}ein, la� es s\X{G}ein, 
       la� es s\X{F}ein, la� es s\X{C}ein.\\
Das Lieblingwort meiner Fr\X{G}au ist: "`La� es s\X{F}ein!"'\X{C dm7 C}

\Xvspace

Darauf will ich Schokolade, gleich zwei, drei Tafeln m�ssen's sein,\\
Meine Frau schon aus der K�che: "`La� es sein!"'\\
Also schnapp ich mir den Playboy, doch kaum f�llt da mein Blick rein,\\
\mbox{}\hspace{1em}---\hspace{1em}erraten:\hspace{1em}---\hspace{1em} 
         "`La� es sein!"'

\textbf{Refrain}

\end{song}











