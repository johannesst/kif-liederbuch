\songtitle{Biddla buh}

\begin{songinfo}
\end{songinfo}

\begin{song}

Es ist traurig, wenn Liebe erkaltet,\\
Es ist furchtbar, wenn Liebe vergeht,\\
Doch wie kann man von Liebe erwarten,\\
Da� sie immer und ewig besteht?\\
Nur ich liebe jede auf immer,\\
Ganz ohne mir das Leben zu erschweren.\\
Und ich werde geliebt. Und wie ich das mach,\\
Das will ich Ihnen jetzt erkl�ren:

Bidla buh, bidla buh, bidla bing, bang, buh,\\
Unsere Liebe war beinahe schon vergangen,\\
Da schlitzte ich die Kehle der Kathrein.\\
Das hei�t, sie liebte mich, solange sie lebte,\\
Und wegen dem bissel Schlitzen wird sie nicht b�se sein.

Bidla buh, bidla buh, bidla bing, bang, buh,\\
Unsere Liebe hatte kaum noch angefangen,\\
Da nahm Jeannine eines Tags ein Aspirin.\\
Das war kein Aspirin, das war Strychnin,\\
Aber heute noch liebe ich Jeannine.

Adelheid warf ich in die Donau,\\
Gleich nach D�rnstein. Niemand hat's gesehn.\\
Und auch sie wird mir verzeihn,\\
Denn grad bei D�rnstein\\
Ist die Donau doch so wundersch�n.

Bidla buh, bidla buh, bidla bing, bang, buh,\\
Also was kann eine Frau da noch verlangen?\\
Nach dem Tod hab ich sie stets noch mehr verehrt.\\
Kam der Tod auch etwas schnell,\\
Das ist nur originell.\\
Und bis jetzt hat sich noch keine beschwert.

Zum Beispiel: Lola, mit den Engelsmienen,\\
Legte ich auf die D-Zug-Schienen,\\
Lilly, Lene und Marianne\\
Starben in der Badewanne,\\
Liesel schlo� den Lebenswandel\\
Durch ein gro�es Ziegelstandel.\\
Lustig ist die J�gerei,\\
Lotte war in Weg dabei.

Buh, bidla buh, bidla bing, bang, buh,\\
Unsere Liebe war kaum �lter als zwei Stunden,\\
Da stieg ich auf den Turm mit Rosemarie.\\
Bei Yvonne hab ich vergessen, den Gashahn abzudrehen,\\
Und die Blumenspenden flossen wie noch nie.

Bidla buh, bidla buh, bidla bing, bang, buh,\\
Nur die Sonja wollte mich versichern lassen.\\
Also, das �rgerte mich sehr.\\
Das hat mich so verdrossen,\\
Ich hab sie schnell erschossen,\\
Und heute lieb ich sie nicht mehr.

Aber Anneliese h�tt' die Krankheit �berwunden,\\
Nur leider trank sie die falsche Arznei.\\
Und Frieda hatte satt das Leben,\\
Wollte selbst den Tod sich geben,\\
Selbstverst�ndlich half ich ihr dabei.

Bidla buh, bidla buh, bidla bing, bang, buh,\\
Aber heute hab ich eine Frau gefunden,\\
Ganz bestimmt die sch�nste Frau der Welt.\\
Und jetzt darf ich's nicht verpassen,\\
Mir das Messer schleifen z'lassen,\\
Und dann mu� ich die Pistolen\\
Vom Pistolenputzer holen,\\
Eine Sense mu� ich borgen,\\
Das Arsen, das kommt erst morgen,\\
Und ein kleines Tomahackerl.\\
F�r die Leich brauch ich ein Sackerl.\\
Auch ein' Besen h�tt' ich gern,\\
Um die Knochen aufzukehrn,\\
Das Petroleum, das hab ich schon bestellt.\\
Bidla buh, bidla buh, bidla bing, bang, buh,\\
Sch�ne Frauen kosten sehr viel Geld.

\end{song}
