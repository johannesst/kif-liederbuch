% Ein -*-latex-*- File

\addsec{Dieses Heft}

Dies ist inzwischen schon die zweite (korrigierte) Version dieses
Liederbuchs. Es enth�lt neue Lieder, teils von der KIF in
Kaiserslautern, teils (vermittels Sly) aus der FS Informatik der Uni
Stuttgart. Ansonsten gilt das schon zur ersten Version gesagte:

Dieses Heft ist eine Sammlung der Lieder, die auf KIF%
\footnote{Konferenz der (deutschsprachigen) Informatikfachschaften,\\
\texttt{\mbox{http://www.informatik.uni-kl.de/fsinform/KIF/}}}s,
auf KIF-Parties, f�r KIFs und um KIFs herum entstanden sind und/oder
gespielt werden. Im ersten Teil finden sich selbstgeschriebene Lieder,
der zweite Teil wird Lieder enthalten, die wir h�ufig spielen, wie das
\emph{Taubenvergiften} oder die \emph{Schmuddelkinder}, und gerne
nehmen wir auch weitere Lieder auf, sofern sich jemand findet, der sie
tippt und setzt.

Insbesondere f�r den zweiten Teil erhoffe ich mir also tatkr�ftige
sowie mannigfaltige Unterst�tzung.%
\footnote{ Ich werde die \LaTeX-Sourcen unter
  \texttt{\mbox{http://www.fachschaften.tu-muenchen.de/\~{}steink/KIF/}}
  zur Verf�gung stellen.}  
Ich werde als Sammelstelle fungieren und jede/r kann sich dann seine
pers�nliche Version des Liederb�ches zusammenstellen und ausdrucken,
bei gro�er Nachfrage w�rde ich auch eine umfassende Version drucken.
Selbstverst�ndlich sind auch Korrekturen etc. erw�nscht.

Dieses Liederbuch kann auf Wunsch mit oder ohne Akkorde �bersetzt
werden, je nach Bedarf. Wenn ich viel Zeit �brig habe oder sich
jemand anderes findet, wird noch Notensatz mit \texttt{mutex},
\texttt{musictex} oder \texttt{musixtex} integriert.

Die erste Version eines KIF-Liederbuches stammt von Sly. Es trug den
Titel: "`Songs der KIF"', war im November 1994 zur KIF in Chemnitz
fertig und enthielt 9 Lieder.

Das vorliegende Heft entstand aufgrund vielf�ltiger Nachfrage nach
Liedtexten auf diversen KIFs und Parties und der unvorsichtigen
Antwort, da� es ein KIF-Liederbuch geben werde, sowie der Arbeit von
KaiN, der einen Teil der Texte getippt hat, und Ariane, die mir beim
Tippen der restlichen Lieder sowie dem Satz geholfen hat.

\mbox{}\hfill Stony\hspace{-0.8mm}\raisebox{-0.7ex}{{\tiny 42}}



\vfill

\begin{quote}
  "`Ich bin vielleicht bl�d, aber daf�r bin ich sch�n!"'
  \mbox{}\hfill\textit{(Averell Dalton)}
\end{quote}


\vfill


